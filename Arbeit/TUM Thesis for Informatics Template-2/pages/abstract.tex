\chapter{\abstractname}

%TODO: Abstract
In order to remain competitive in the ongoing globalization, companies are forced to optimize their productions and processes. The immense amount of data and modern information and communication technologies offer great opportunities for modern prognostics and health management systems. By analyzing machine and production data, machines can be proactively maintained, which increases the quality standards and efficiency in production lines. Since ball screw feed drives are widely used components in industrial machines, their degradation status monitoring is especially important. Due to varying working conditions, prognostics and health management systems must be developed robust enough to handle continuous changes in the fault characteristics of industrial machines. To counteract the domain shift in machine data, novel domain adaptation approaches from the computer vision community recently became more popular in prognostics and health management systems. In this thesis, a deep learning-based domain adaptation model is proposed, which predicts the health condition state of ball screw feed drives. Throughout the training, the domain shift in the machine data is measured and reduced by the maximum mean discrepancy. By applying the maximum mean discrepancy in the feature extractor, the domain shift is reduced more efficiently. The effects of different hyperparameter choices in the domain adaptation module are evaluated. A novel maximum mean discrepancy metric is presented, which uses some target labels. 


\makeatletter
\ifthenelse{\pdf@strcmp{\languagename}{english}=0}
{\renewcommand{\abstractname}{Kurzfassung}}
{\renewcommand{\abstractname}{Abstract}}
\makeatother

\chapter{\abstractname}

%TODO: Abstract in other language
\begin{otherlanguage}{ngerman} % TODO: select other language, either ngerman or english !
Um in der fortschreitenden Globalisierung wettbewerbsfähig zu bleiben, sind Unternehmen gezwungen, ihre Produktion und Prozesse zu optimieren. Die dabei generierten Datenmengen sowie moderne Informations- und Kommunikationstechniken bieten große Chancen für moderne Prognostics and Health Management Systeme. Durch die Analyse von Maschinen- und Produktionsdaten können Maschinen proaktiv gewartet und dadurch die Qualitätsstandards und Effizienz in Produktionslinien erhöht werden. Kugelgewindetriebe sind häufig in Industriemaschinen verbaut. Deren Zustandsüberwachung is daher besonders wichtig. Aufgrund sich ändernder Arbeitsbedingungen müssen Prognostics and Health Management Systeme entwickelt werden, welche robust genug sind, um mit kontinuierlichen Veränderungen in den Fehlercharakteristiken der Industriemaschinen umzugehen. Um der Domänenverschiebung in den Maschinendaten entgegenzuwirken, wird der Einsatz von Domänenanpassungsansätzen aus der Computer Vision in Prognostics and Health Management Systemen immer beliebter. In dieser Arbeit wird ein Deep Learning basiertes Domänenanpassungssystem vorgestellt, welches den Verschleißzustand von Kugelgewindetrieben vorhersagt. Während des Trainings wird die Domänenverschiebung in den Maschinendaten durch die Maximum Mean Discrepancy gemessen und reduziert. Durch den Einsatz der Maximum Mean Discrepancy in CNNs wird die Domänenverschiebung effizienter reduziert. Die Effekte verschiedener Hyperparameter im Domänenanpassungsmodul werden evaluaiert. Eine neuartige Maximum Mean Discrepancy Metrik wird vorgestellt, welche die Target Labels zu einem kleinen Teil verwendet.



\end{otherlanguage}


% Undo the name switch
\makeatletter
\ifthenelse{\pdf@strcmp{\languagename}{english}=0}
{\renewcommand{\abstractname}{Abstract}}
{\renewcommand{\abstractname}{Kurzfassung}}
\makeatother