\chapter{\abstractname}

%TODO: Abstract
In order to remain competitive in the ongoing globalization, companies are forced to optimize their productions and processes. The immense amount of data and modern technologies offers great opportunities for modern Predictive Maintenance systems. By analysing machine and production data, machines can be proactively maintained, which increases quality standards and productivity. Ball screw drives are a common part in industrial machines, their health monitoring is especially interesting. Due to changing working conditions, prognostics and health management systems must be developed robust enough to handle continuous changes in the fault characteristics of industrial machines. To counteract the domain shift in the machine data, novel domain adaption approaches from the computer vision community can be applied. In this thesis a deep learning based domain adaption system was developed to predict the health condition state of ball screw drives. The domain shift in the machine data is measured by the maximum mean discrepancy and reduced throughout the model training.

\makeatletter
\ifthenelse{\pdf@strcmp{\languagename}{english}=0}
{\renewcommand{\abstractname}{Kurzfassung}}
{\renewcommand{\abstractname}{Abstract}}
\makeatother

\chapter{\abstractname}

%TODO: Abstract in other language
\begin{otherlanguage}{ngerman} % TODO: select other language, either ngerman or english !
Um in der fortschreitenden Globalisierung wettbewerbsfähig zu bleiben, sind Unternehmen gezwungen, ihre Produktionen und Prozesse zu optimieren. Die dabei generierten Datenmengen und die modernen Technologien bieten große Chancen für moderne Predictive Maintenance Systeme. Durch die Analyse von Maschinen- und Produktionsdaten können Maschinen proaktiv gewartet werden, um dadurch die Qualitätsstandards und Produktivität zu erhöhen. Kugelgewindetriebe sind häufig in Industriemaschinen verbaut. Dadurch ist deren Zustandsüberwachung besonders interessant. Aufgrund sich ändernder Arbeitsbedingungen müssen Predictive Maintenance Systeme entwickelt werden, die robust genug sind, um mit kontinuierlichen Änderungen der Fehlercharakteristiken von Industriemaschinen umzugehen. Um der Domänendiskrepanz in den Maschinendaten entgegenzuwirken, können neuartige Domänenanpassungsansätze aus der Computer Vision Community eingesetzt werden. In dieser Arbeit wurde ein auf Deep Learning basierendes Domänenadaptionssystem entwickelt, um den Verschleißzustand von Kugelgewindetrieben vorherzusagen. Die Domänenverschiebung in den Maschinendaten wird durch die Maximum Mean Discrepancy gemessen und während des Modelltrainings reduziert.


\end{otherlanguage}


% Undo the name switch
\makeatletter
\ifthenelse{\pdf@strcmp{\languagename}{english}=0}
{\renewcommand{\abstractname}{Abstract}}
{\renewcommand{\abstractname}{Kurzfassung}}
\makeatother