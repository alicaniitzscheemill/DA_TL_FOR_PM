\chapter{\abstractname}

%TODO: Abstract
In order to remain competitive in the ongoing globalization, companies are forced to optimize their productions and processes. The immense amount of data, as well as modern information and communication technologies offer great opportunities for modern prognostics and health management. By analyzing machine and production data, machines can be proactively maintained, increasing the quality standards and efficiency of production lines. Since ball screw feed drives are widely used components in industrial machines, their degradation monitoring is especially important. Due to varying working conditions, prognostics and health management systems must be developed robust enough to handle continuous changes in the fault characteristics of industrial machines. Modern domain adaptation approaches have recently become more popular in prognostics and health management systems to counteract the corresponding domain shift in the machine data. This thesis proposes a deep learning-based domain adaptation model, which predicts the health condition of ball screw feed drives. Throughout the proposed model training, the domain shift in the latent feature spaces of the model is measured and reduced by the maximum mean discrepancy. By applying the maximum mean discrepancy in the CNN layers, the domain shift is reduced more efficiently. The effects of different hyperparameter choices in the domain adaptation module are evaluated. A novel labeled maximum mean discrepancy metric is presented, which uses a small portion of the training target labels. 


\makeatletter
\ifthenelse{\pdf@strcmp{\languagename}{english}=0}
{\renewcommand{\abstractname}{Kurzfassung}}
{\renewcommand{\abstractname}{Abstract}}
\makeatother

\chapter{\abstractname}

%TODO: Abstract in other language
\begin{otherlanguage}{ngerman} % TODO: select other language, either ngerman or english !
Um in der fortschreitenden Globalisierung wettbewerbsfähig zu bleiben, sind Unternehmen gezwungen, ihre Produktion und Prozesse zu optimieren. Die dabei generierten Datenmengen sowie moderne Informations- und Kommunikationstechniken bieten große Chancen für moderne Prognostics and Health Management Systeme. Durch die Analyse von Maschinen- und Produktionsdaten können Maschinen proaktiv gewartet und dadurch die Qualitätsstandards und Effizienz von Produktionslinien erhöht werden. Da Kugelgewindetriebe häufig in Industriemaschinen verbaut sind, ist deren Zustandsüberwachung besonders wichtig. Aufgrund sich ändernder Arbeitsbedingungen müssen Prognostics and Health Management Systeme entwickelt werden, welche robust genug sind, um mit kontinuierlichen Veränderungen in den Fehlercharakteristiken der Industriemaschinen umzugehen. Um dem damit einhergehenden Domain Shift entgegenzuwirken, wird der Einsatz von Domain Adaptation Ansätzen, welche ursprünglich aus dem Bereich Maschinelles Sehen kommen, in Prognostics and Health Management Systemen immer beliebter. In dieser Arbeit wird ein Deep Learning basiertes Domain Adaptation System vorgestellt, welches den Verschleißzustand von Kugelgewindetrieben vorhersagt. In dem erarbeiteten Modelltraining wird der Domain Shift in den Schichten des Modells durch die Maximum Mean Discrepancy gemessen und reduziert. Durch den Einsatz der Maximum Mean Discrepancy in CNN Schichten, wird der Domain Shift effizient reduziert. Die Effekte verschiedener Hyperparameter im Domain Adaptation Modul werden evaluiert. Eine neuartige Labeled Maximum Mean Discrepancy Metrik wird vorgestellt, welche zu einem kleinen Teil Target Labels verwendet.



\end{otherlanguage}


% Undo the name switch
\makeatletter
\ifthenelse{\pdf@strcmp{\languagename}{english}=0}
{\renewcommand{\abstractname}{Abstract}}
{\renewcommand{\abstractname}{Kurzfassung}}
\makeatother