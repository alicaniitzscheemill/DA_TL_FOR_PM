\chapter{Outlook}
There are still several open and interesting topics, which are worth to be investigated for the PHM of BSDs. GANs were already investigated for the PHM of other industrial parts. Zhang et al \cite{Zhang2019} developed a wasserstein distance guided multi-adversarial network for the prediction of the degradation status of rolling bearing. The development and training of such methods might be more complex. Anyhow, the adversarial training is especially promising for the extraction of domain-invariant features \cite{Zhang2019}. A less complex method, which uses similar mechanisms as GANs was presented by Guo et al \cite{Guo2019}. They developed a deep learning based model which reduces the domain discrepancy based on a MMD-loss. In addition a domain classifier is included, which tries to predict the domain for each processed sample. The feature extractor is optimized based on a weighted loss, which minimizes the source CE- and MMD-loss and maximizes the domain discriminator loss. Also the preprocessing of the recorded machine signals could be improved. In this thesis a simple windowing function was used to separate the data in shorter sequences.Generally, the generated windows should capture the degradation related patterns of the vibration signals. For this reason, the windows should be adjusted to the consistency and periodicity of the data. The windowing requirements differ for the machine excitements (constant speed excitement, direction change excitement and sweep excitement). Generally, when choosing the window size, there is a trade-off between the window size and the number of windows generated from the data. Both extremes (few big windows and numerous small windows) might lead to problems during the training. The PHM results might be improved by applying an adaptive preprocessing to generate data windows of suitable length, which are well synchronized with the data. Lastly, also thepotential performance gains due to the combination of several signals could be investigated in more detail.