\chapter{Outlook}
In future work, the dataloader could be extended with more sophisticated pre-processing steps. Wavelet transforms or FFTs, could be applied to the raw data to feed more expressive data to the neural network. In the literature, it is quite common to include such steps. Investigating to what extent such methods could increase the PHM performance would be very interesting.
In the experiments of this thesis, the best PHM results were achieved with the direction change excitation and the worst with the sweep excitation. The window size was set to 1024 and the windows were not synchronized with the data. In the direction change excitation signals, the windows could capture several periods of direction change movements. In the constant speed excitation signals, the windows were generally large enough to capture all data points from the direction change phase and the constant velocity phase but the windows were not properly synchronized with those. In the sweep excitation signals, the windows could not cover all corresponding data points. The excitation and thus the signals showing the machine's reaction to those differ fundamentally in their periodicity and consistency. The dependence of the PHM performance on the excitation might be caused by different window requirements to capture the corresponding underlying degradation patterns. The PHM performance might be improved by adapting and synchronizing the windows to the periodicity and consistency of the excitation. In this case, the machine's reaction to the different excitation can then be better observed and evaluated. The amplitude and frequency variations during one excitation period or a constant BSD velocity phase might deliver expressive information about the machine's degradation. In the literature, applying regime separation to train the models only on data recorded during phases of constant BSD velocity is very common.
Furthermore, even more realistic domain shift problems could be generated to test the proposed PHM system. Training and testing the model on differently degraded LGSs would be an realistic and challenging task.
There are several other promising approaches that are worth to be investigated for the PHM of BSDs. Recently, GANs became more popular for monitoring the health condition of different industrial parts \cite{Zhang2019}. However, the development and training of such methods might be more complex. Especially when applying such models to noisy and disturbed real-world data, this might lead to instability during the training. However, the adversarial training is especially promising for the extraction of domain-invariant features \cite{Zhang2019}. 




\begin{comment}
Also the preprocessing of the recorded machine signals could be improved. In this thesis a simple windowing function was used to separate the data in shorter sequences. Generally, the generated windows should capture the degradation related patterns of the vibration signals. For this reason, the windows should be adjusted to the consistency and periodicity of the data. The windowing requirements differ for the machine excitements (constant speed excitement, direction change excitement and sweep excitement). Generally, when choosing the window size, there is a trade-off between the window size and the number of windows generated from the data. Both extremes (few big windows and numerous small windows) might lead to problems during the training. The PHM results might be improved by applying an adaptive preprocessing to generate data windows of suitable length, which are well synchronized with the data. Lastly, also the potential performance gains due to the combination of several signals could be investigated in more detail.
\end{comment}