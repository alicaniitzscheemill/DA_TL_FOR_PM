\chapter{Research Approach}\label{chapter:research_approach}

When industrial machines run over long time horizons, operational conditions and therefore fault characteristics might change. Reasons for that can be abrasion, changed machine settings or minor installation differences when submodules of the machine were replaced. Due to the correlation and dependency between different machine submodules, fault characteristics often don't show clear patterns. Developing hand-crafted features, as they are quite common in the traditional approaches, expect a lot of experience and human labor. Due to the lack of flexibility and robustness, hand-crafted features will most certainly fail to extract expressive information from the data when fault characteristics change. Model-based PHM systems, as the one presented by Lee et al \cite{Lee2015}, expect a big amount of human labor and experience to be tuned properly. Li et al calculate defect frequencies, which are characteristic for specific faults. With varying fault characteristics approaches like that might most probably fail. Data-driven based PHM systems, like the one proposed by Denkena et al \cite{Denkena2021}, might find more general and expressive features, which are less prone to small variations in the fault characteristics. Anyhow, when these variations become too big, the prediction accuracy will decrease as well. The performance of such systems depends a lot on the underlying training dataset. It is unlikely that the data used for training includes all operational conditions and fault scenarios. It can even happen that fault classes are unknown during training. Training neural networks on limited amount of data, which does not represent the data distributions during testing, might lead to unsatisfactory diagnosis performance while testing \cite{AZAMFAR2020103932}. Robust PHM systems, which can handle variations in fault characteristics or even unseen fault classes, would bring industrial PHM systems to a next level \cite{Michau2017}. In order to address those issues, domain adaption approaches seem promising. This thesis investigates the applicability and usability of deep learning based PHM systems using domain adaption for degradation estimation in BSDs. The advantages of the proposed systems over regular deep learning based systems is evaluated. Since it expects a lot of work to establish accurate model-based PHM approaches, the developed approaches in this thesis could not be compared with such systems.

\section{Research Questions}
The thesis is centered around three main research questions, which are presented in the following. These questions were defined beforehand and should make sure that the developed PHM system is analyzed properly. Mainly the questions were formulated based on common problems and challenges formulated by the PHM community. 

\subsection{Influence of latent feature space choice on the domain adaption performance}
Most domain adaption approaches, just as the one presented by Azamfar et al \cite{AZAMFAR2020103932} and Pandhare et al \cite{Pandhare2021}, reduce the domain discrepancy in task-specific layers but use a shared feature extractor backbone across all domains. Li et al \cite{li2020} assume that, if the domain discrepancy is tremendously large, these methods can only reduce the domain discrepancy, but not fundamentally eliminate it. For this reason, this thesis investigates how applying the MMD loss in the feature extractor can help to reduce the discrepancy more efficiently challenging tasks.

\subsection{Influence of GAMMA choice on the domain adaption performance}
Since the source and target domains are correlated to some extend, the network itself can extract domain-independent features. The powerful feature extractor learned from the source domain can also increase the model performance on the target domain. At the same time, features which are too sensitive to the source domain can reduce the model performance on the target domain \cite{li2020}. To counteract that phenomena, domain adaption approaches can help to transfer knowledge learned on the source to the target domain. Anyhow, one has to pay attention to not transfer noise or irrelevant information, since this destroys the structure of the source and target domain data and makes the classification task even more difficult \cite{li2020}. It is important to balance the MMD- and CE-loss very sensible. In this thesis the effects of different GAMMA choices are investigated.

\subsection{Domain adaption performance when using a labeled MMD-loss}
Pandhare et al \cite{Pandhare2021} apply PD-alignment, which specifically reduces the L2-distance between samples belonging to the same class but different domains. This increases the domain overlap in the latent feature space. With an increasing domain similarity, the extraction of domain invariant features becomes easier. Unfortunately, the labels of the source and target domain samples need to be known to apply the PD-alignment. In theory the positive effects of PD-alignment are obvious. This thesis analyzes how target labels can improve the MMD-based domain adaption capabilities. Instead of applying a PD-alignment in addition to a MMD-loss a novel MMD-loss which considers the classes of source and target domain samples was developed. Like in the work of Pandhare et al \cite{Pandhare2021}, the target labels are not used in the CE-loss. Since it is difficult to compare PHM systems which have access to different data, the PHM approaches evaluated on the real-world dataset is restricted those, which only use source domain labels. Li et al \cite{Li2018} present a PHM algorithm for rolling bearings, which optimizes the inter- and intra-class distance while reducing the domain discrepancy with a MMD-loss. For all samples belonging to one class the expectation and variance is measured in the feature maps of interest. The expectation and variance are used to optimize the intra- and inter-class distances for the source domain samples. This approach by Li et al increases the compactness and separability of the source domain samples in the latent feature spaces. Since the distance-based optimization is restricted to the source domain, it's effectiveness for a domain adaption task is questionable.

\section{Signals used for PHM}
In the work of Pandhare et al \cite{Pandhare2021} just vibration signals in different spatial directions are measured with sensors, installed at various locations on the BSD. Azamfar et al \cite{AZAMFAR2020103932} additionally use sound pressure sensors to capture the acoustic level and extract torque and speed signals from the controller. In this thesis also the suitability of the mechanical power, target electrical power and actual electrical power signals were extracted from the controller and evaluated for the PHM on BSDs. Pandhare et al and Azamfar et al record machine data during BSD steady-state motion. In this thesis machine data is collected during different machine excitements (constant speed excitements, direction change excitements and sweep excitement) along the machine tools X-axis.

\begin{comment}
Both Pandhare et al \cite{Pandhare2021} and Azamfar et al \cite{AZAMFAR2020103932} feed the data recorded during BSD steady-state motion as one single input to their models. During the phases of constant BSD motion, the amplitude of the signals changess. Azamfar et al assume that the shorter sequences created by a windowing function just capture limited information about these changes and are therefore not a proper tool for PHM \cite{AZAMFAR2020103932}. In the thesis a windowing function was evaluated for the PHM of BSDs. Windowing functions make the BSD experiments less dependent from specific BSD excitements. When beeing able to check the BSD degradation with short recorded windows, one can make statements about the BSD health status with data redcorded in real time use. Extra experiments 
\end{comment}

\section{PHM for rolling bearing}
There are several domain adaption approaches for PHM of rolling bearing based on MMD-losses \cite{Guo2019} \cite{Singh2019} \cite{Li2018} \cite{AN201942} \cite{Kang2020}. Generally, BSDs and rolling bearing are related components. The BSD shaft can be seen as the inner ring and the BSD nut as the outer ring of a rolling bearing. In both cases balls create a movable bearing between those two parts, such that a rotatory motion along a fixed axis is enabled. Other than rolling bearings, BSDs also translate this rotatory motion in a linear motion between BSD shaft and nut. The degradation of rolling bearings and BSDs is related in some sense. Nevertheless, bearing PHM applications cannot be relied to work well for BSDs. Still the research in this domain offers interesting applications and details for BSD PHM. 

\section{Other deep-learning based domain adaption approaches for PHM}
Besides that, there are also quite interesting applications which use multi-adversarial networks \cite{Zhang2019} or deep belief networks (DBN) \cite{ZHAO2019213} for PHM applications. When training general adversarial neural networks (GANs), two networks, which work against each other, need to be optimized simultaneously. Deep belief networks (DBN) contain several stacked restricted Boltzmann machines (RBMs). The training of DBNs is separated in two phases. First, all RBMs are optimized individually. Afterwards, the DBN is fine-tuned to solve the classification task by applying backpropagation on all RBM layers simultaneously. Especially when applying such networks on noisy and disturbed real-world vibration signals, this might lead to instabilities. For this reason and due to the limited time of this thesis, the MMD-based domain discrepancy reduction is the focus of this thesis.
