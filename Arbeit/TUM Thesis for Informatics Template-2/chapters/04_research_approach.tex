\chapter{Research Questions}\label{chapter:research_approach}
The thesis is centered around three main research questions, which are presented in the following. These research questions were defined beforehand to guarantee a structured development of the PHM systems. The questions were formulated based on common problems and challenges in the domain adaptation and the PHM community. At the end of this chapter, the research approach followed in this thesis is described.

\section{Influence of the GAMMA Choice on the Domain Adaptation Performance}
Since the source and target domains are correlated to some extent, the network itself can extract domain-independent features. The powerful CNN learned from the source domain can also increase the model performance on the target domain \cite{li2020}. At the same time, features that are too sensitive to the source domain can reduce the model performance on the target domain \cite{li2020}. To counteract that phenomenon, domain adaptation approaches can help to transfer knowledge learned from the source to the target domain \cite{li2020}. However, one has to pay attention to not transfer noise or irrelevant information since this destroys the structure of the source and target domain data and makes the classification task even more difficult \cite{li2020}. For this reason, it is essential to precisely balance the MMD- and source CE-loss. This thesis investigates the effects of different weighting factors, called GAMMA, on the model training.

\section{Domain Adaptation Performance of the Labeled MMD-Loss}
In addition to the MMD-based domain discrepancy reduction, Pandhare et al. \cite{Pandhare2021} applied a PD alignment module, which specifically reduced the L2-distance between samples belonging to the same class but different domains. The PD alignment required target labels. Li et al. \cite{Li2018} presented a PHM algorithm, which optimized the source domain inter- and intra-class distance while reducing the domain discrepancy with an MMD-loss. Both approaches try to improve the classes' compactness and separability in the latent feature spaces while reducing the domain discrepancy. This enables a more accurate classification and improves the domain overlap in the latent feature space. A stronger domain similarity facilitates the extraction of domain-invariant features. Inspired by those two approaches, this thesis developed a novel labeled MMD-loss, which minimizes the domain discrepancy between samples of the same class and maximizes the domain discrepancy between samples of different classes. Instead of applying a PD alignment in addition to an MMD-loss, this novel MMD-loss directly considers the labels of the source and target domain. Similar to the work of Pandhare et al. \cite{Pandhare2021}, the target labels are not used in the CE-loss. The advantages and disadvantages of the labeled MMD-loss over the unlabeled MMD-loss are further analyzed in this thesis.

\section{Influence of the Latent Feature Space Choice on the Domain Adaptation Performance}
Most domain adaptation approaches, just as the one presented by Azamfar et al. \cite{AZAMFAR2020103932} and Pandhare et al. \cite{Pandhare2021}, reduce the domain discrepancy in the task-specific layers and use a shared CNN backbone across all domains. Throughout the neural network, the feature maps extract information with different levels of abstraction. In shallow, more global, and deeper layers, more task-specific features are extracted \cite{Aljundi2016}. Often, it is assumed that early convolutional layers extract general information and act as simple detectors for things like color or texture \cite{Aljundi2016}. However, the work of Aljundi et al. \cite{Aljundi2016} shows that the layers of the CNN are responsible for specific characteristics in deeper latent feature space representations and thus the creation of a dataset bias. Since feature maps influence all subsequent ones, propagating the biased data through the neural network facilitates the domain shift \cite{Aljundi2016}. Aljundi et al. \cite{Aljundi2016} showed that the CNN layers significantly contribute to the increasing domain shift throughout the neural network. Aljundi et al. \cite{Aljundi2016} mentioned that there is no clear pattern to which extent layers contribute to and suffer from the domain shift. Reducing the domain discrepancy in only task-specific layers might minimize but not completely eliminate it \cite{li2020}. Inspired by the work of Aljundi et al. \cite{Aljundi2016}, this thesis investigates how applying the MMD loss in different model layers can improve the domain discrepancy reduction. In this context, a particular focus lies on the layers of the CNN.

\section{Research Approach}
This thesis investigates the ability of domain adaptation approaches to increase the performance of PHM systems. First, the developed approaches were pre-evaluated on a dummy dataset and afterwards tested on a real-world dataset. In the research community, simplified synthetically generated datasets are quite common to understand the underlying functionalities and mechanisms of new approaches and to pre-evaluate their applicability for a given task. For the sake of completeness, it is necessary to test these approaches on real-world data. When this has been done correctly, statements about the approaches' robustness, applicability and utility can be made. Experiments on the mentioned datasets were performed to investigate the following topics:
\begin{itemize}
    \item Visualization of the influence of MMD-based domain adaptation on the latent feature space representations of neural networks.
    \item Evaluation of the PHM performance gain due to MMD-based domain adaptation
    \item Investigation of the three formulated research questions in the context of MMD-based domain adaptation in industrial health condition monitoring.
\end{itemize}

The datasets, proposed model architecture and corresponding training strategy are described in chapter \ref{sec:experiments}. The results from the experiments are presented in chapter \ref{sec:results}. 
\begin{comment}
\section{Signals used for PHM}
In the work of Pandhare et al. \cite{Pandhare2021} just vibration signals in different spatial directions are measured with sensors, installed at various locations on the BSD. Azamfar et al. \cite{AZAMFAR2020103932} additionally use sound pressure sensors to capture the acoustic level and extract torque and speed signals from the controller. In this thesis also the mechanical power, target electrical power and actual electrical power signals were extracted from the controller. Pandhare et al. and Azamfar et al. record machine data during BSD steady-state motion. In this thesis machine data is collected during different machine excitements (constant speed excitements, direction change excitements and sweep excitement) along the machine tools X-axis. These different signals were evaluated for their suitability for PHM of BSDs


Both Pandhare et al. \cite{Pandhare2021} and Azamfar et al. \cite{AZAMFAR2020103932} feed the data recorded during BSD steady-state motion as one single input to their models. During the phases of constant BSD motion, the amplitude of the signals changess. Azamfar et al. assume that the shorter sequences created by a windowing function just capture limited information about these changes and are therefore not a proper tool for PHM \cite{AZAMFAR2020103932}. In the thesis a windowing function was evaluated for the PHM of BSDs. Windowing functions make the BSD experiments less dependent from specific BSD excitements. When beeing able to check the BSD degradation with short recorded windows, one can make statements about the BSD health status with data redcorded in real time use. Extra experiments 
\end{comment}


