\chapter{Research Approach}\label{chapter:research_approach}

When industrial machines run over long time horizons, operational conditions and therefore fault characteristics might change. Reasons for that can be abrasion, changed machine settings or minor installation differences after replacing submodules of the machine. Due to the interaction of different machine submodules, fault characteristics are often complex and highly dependent. Developing hand-crafted features, as they are quite common in the traditional approaches, expect a lot of experience and human labor. Due to the lack of flexibility and robustness, hand-crafted features struggle to extract expressive information from the data with changing fault characteristics. The quality of model-based PHM systems highly depends on the exactness and sophistication of the underlying modeling and the corresponding parameters, which are found through experiments. Li et al calculate BSD defect frequencies, based on a simplified model. The defect frequencies of BSDs might be influenced by other machine modules, which are not considered in this frequency estimation. With changing operational conditions these characteristic frequencies might also change to some extent. When not considering such varying fault characteristics in the model-based approaches their PHM performance will decrease over time. Data-driven PHM systems, like the one proposed by Denkena et al \cite{Denkena2021}, might find more general and expressive features, which are less prone to small variations in the fault characteristics. The performance of such systems depends a lot on the underlying training dataset. It is unlikely that the data used for training includes all operational conditions and fault scenarios. It can even happen that fault classes are unknown during training. Training neural networks on limited amount of data, which does not represent the data distributions during testing, might lead to unsatisfactory diagnosis performance while testing \cite{AZAMFAR2020103932}. Robust PHM systems, which can handle variations in fault characteristics, would bring industrial PHM systems to a next level \cite{Michau2017}. In order to address those issues, domain adaption approaches seem promising. This thesis investigates the applicability and usability of deep learning-based PHM systems using domain adaption for health monitoring of BSDs. The advantages of the proposed systems over regular deep learning-based systems are evaluated. Since it expects a lot of work to establish accurate model-based PHM approaches, the developed approaches in this thesis could not be compared with such systems.

\section{Research Questions}
The thesis is centered around three main research questions, which are presented in the following. These questions were defined beforehand and should make sure that the developed PHM system is analyzed properly. Mainly the questions were formulated based on common problems and challenges formulated by the PHM community. 

\subsection{Influence of latent feature space choice on the domain adaption performance}
Most domain adaption approaches, just as the one presented by Azamfar et al \cite{AZAMFAR2020103932} and Pandhare et al \cite{Pandhare2021}, reduce the domain discrepancy in task-specific layers but use a shared feature extractor backbone across all domains. Li et al \cite{li2020} assume that, if the domain discrepancy is tremendously large, these methods can only reduce the domain discrepancy, but not fundamentally eliminate it. For this reason, they propose a model which learns to adaptively extract domain-dependent and -independent features in the convolutional layers to reduce the domain shift. Still, Li et al rely on the positive effect of additionally reduce the domain discrepancy in deeper task-specific layers with a MMD-loss \cite{li2020}. Throughout a neural network feature maps with different abstraction layers are extracted. In shallow layers more global and in deeper layers more task-specific features are extracted. Generally, people assume that early convolutional layers just extract general information and act as simple detector for things like colour or texture. The work of Aljundi et al show, that those layers are responsible for specific characteristics in the latent feature map representation of the data and thus the dataset bias. Since feature maps influence all subsequent ones, propagating the biased data through the neural network facilitates the domain shift in the final layers of a neural network. Therefore, Aljundi et al suggest to evaluate the domain shift in early convolutional layer and reconstruct the layer if necessary \cite{Aljundi2016}. Inspired by the work of Azamfar et al and Aljundi et al, this thesis investigates how applying the MMD loss in the feature extractor can improve the domain discrepancy reduction.








\subsection{Influence of GAMMA choice on the domain adaption performance}
Since the source and target domains are correlated to some extend, the network itself can extract domain-independent features. The powerful feature extractor learned from the source domain can also increase the model performance on the target domain. At the same time, features which are too sensitive to the source domain can reduce the model performance on the target domain \cite{li2020}. To counteract that phenomena, domain adaption approaches can help to transfer knowledge learned from the source to the target domain. Anyhow, one has to pay attention to not transfer noise or irrelevant information, since this destroys the structure of the source and target domain data and makes the classification task even more difficult \cite{li2020}. It is important to balance the MMD- and CE-loss very sensible. In this thesis the effects of different GAMMA choices are investigated.

\subsection{Domain adaption performance when using a labeled MMD-loss}
Pandhare et al \cite{Pandhare2021} include PD-alignment in their model training, which specifically reduces the L2-distance between samples belonging to the same class but different domains. This increases the domain overlap in the latent feature space. With an increasing domain similarity, the extraction of domain invariant features becomes easier. Unfortunately, the PD-alignment expects target and source labels. This thesis analyzes how target labels can improve the MMD-based domain adaption capabilities. Instead of applying a PD-alignment in addition to a MMD-loss a novel MMD-loss which considers the labels of source and target domain samples was developed. Like in the work of Pandhare et al \cite{Pandhare2021}, the target labels are not used in the CE-loss. Comparing PHM systems which have access to different data is difficult. For this reason, just those PHM approaches are evaluated on the real-world dataset, which solely have access to source domain labels.  Besides that, the distance-based optimization modules require at least one additional hyperparameter to GAMMA, which makes the tuning of the model even more difficult. Li et al \cite{Li2018} present a PHM algorithm for rolling bearings, which optimizes the source domain inter- and intra-class distance while reducing the domain discrepancy with a MMD-loss. For all samples belonging to the same class the expectation and variance is measured in the feature maps of interest. The expectation and variance are used to optimize the intra- and inter-class distances for the source domain samples. This approach by Li et al increases the compactness and separability of the source domain samples in the latent feature spaces. Since the distance-based optimization is restricted to the source domain, it's effectiveness for a domain adaption task is questionable.

\section{Signals used for PHM}
In the work of Pandhare et al \cite{Pandhare2021} just vibration signals in different spatial directions are measured with sensors, installed at various locations on the BSD. Azamfar et al \cite{AZAMFAR2020103932} additionally use sound pressure sensors to capture the acoustic level and extract torque and speed signals from the controller. In this thesis also the mechanical power, target electrical power and actual electrical power signals were extracted from the controller. Pandhare et al and Azamfar et al record machine data during BSD steady-state motion. In this thesis machine data is collected during different machine excitements (constant speed excitements, direction change excitements and sweep excitement) along the machine tools X-axis. These different signals were evaluated for their suitability for PHM of BSDs

\begin{comment}
Both Pandhare et al \cite{Pandhare2021} and Azamfar et al \cite{AZAMFAR2020103932} feed the data recorded during BSD steady-state motion as one single input to their models. During the phases of constant BSD motion, the amplitude of the signals changess. Azamfar et al assume that the shorter sequences created by a windowing function just capture limited information about these changes and are therefore not a proper tool for PHM \cite{AZAMFAR2020103932}. In the thesis a windowing function was evaluated for the PHM of BSDs. Windowing functions make the BSD experiments less dependent from specific BSD excitements. When beeing able to check the BSD degradation with short recorded windows, one can make statements about the BSD health status with data redcorded in real time use. Extra experiments 
\end{comment}

\section{PHM for rolling bearing}
There are several domain adaption approaches for PHM of rolling bearing based on MMD-losses \cite{AN201942} \cite{Li2018} \cite{Guo2019} \cite{Singh2019} \cite{Kang2020}. Generally, BSDs and rolling bearing are related components. BSD shafts can be seen as the inner and BSD nuts as the outer ring of rolling bearings. In both cases balls between those two components allow a rotatory motion around a fix axis. Other than rolling bearings, BSDs also translate this rotatory motion in a linear motion. The degradation of rolling bearings and BSDs is related in some sense. Nevertheless, bearing PHM applications cannot be relied to work as well for BSDs. Still the research in this domain offers interesting applications and details for the PHM of BSDs. 

\section{Other deep learning-based domain adaption approaches for PHM}
Besides that, there are also quite interesting applications which use multi-adversarial networks \cite{Zhang2019} or deep belief networks (DBN) \cite{ZHAO2019213} for PHM applications. When training general adversarial neural networks (GANs), two networks, which work against each other, need to be optimized simultaneously. Deep belief networks (DBN) contain several stacked restricted Boltzmann machines (RBMs). The training of DBNs is separated in two phases. First, all RBMs are optimized individually. Afterwards, the DBN is fine-tuned to solve the classification task by applying backpropagation on all RBM layers simultaneously. Especially when applying such networks on noisy and disturbed real-world vibration signals, this might lead to training instabilities. For this reason and due to the limited time of this thesis, the MMD-based domain discrepancy reduction is the focus of this thesis.
