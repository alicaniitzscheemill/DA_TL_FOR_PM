\chapter{Conclusion}\label{chapter:conclusion}
In this thesis a deep learning-based PHM systems for BSDs was successfully extended with a domain adaption module. A novel way of reducing the domain discrepancy in the network's feature extractor with a MMD-loss was developed. This approach outperforms the more traditional approaches of applying the MMD-loss in the classifier layers. The experiments showed that the GAMMA choice is very relevant for the PHM performance and has to be picked carefully and individually for each signal. A novel labeled MMD-loss, which considers the soure and target labels was presented. Depending on the class label of the source and target sample, the labeled MMD-loss increases or decreases the domain discrepancy. The labeled MMD-loss was not investigated on the real-world dataset but showed promising results in the evaluation on the dummy dataset. The developed domain adaption approaches were compared with a base line model relying solely on the capabilities of neural networks to solve the PHM task.
