\chapter{Conclusion}\label{chapter:conclusion}

In this thesis, a deep learning-based PHM system for monitoring BSDs was successfully extended with a domain adaptation module. A novel way of applying the MMD-loss in the layers of the CNN was proposed, which reduced the domain discrepancy efficiently. The proposed MMD-loss outperformed those restricted to the task-specific layers, which were mainly used in the literature. This was mainly reflected in the increased prediction accuracy and training stability but also in the reduced sensibility to different signal and GAMMA choices. The development of a novel labeled MMD-loss revealed the main deficit of the regular unlabeled MMD-loss, which equally reduces the domain discrepancy between samples of the same and different classes. If the unlabeled MMD-loss became too dominant, the class separability in both domains was reduced, which made the classification task more challenging. Only when balancing the source CE- and MMD-loss properly the domain discrepancy was reduced while maintaining or increasing the compactness and separability of the classes. For this reason, the GAMMA choice was identified as highly relevant for the PHM performance. It is suggested to pick the GAMMA precisely and individually for each signal. The developed models were evaluated on a dataset recorded on an industrial machine. Therefore, the mutual influence of different components of the machine was considered in the model evaluation. Also, the degradation of the BSDs and LGSs was caused by field use. Compared to the presented works in chapter \ref{chapter:related_works}, the evaluation of the developed PHM system was more realistic and elaborated in this thesis. Independent of the LGS degradation, the developed model was able to predict the health condition of the BSDs. Using different sets of BSDs in the source and target domain created a domain shift on the physical component level. Since the preload class definitions differed for both domains, the domain shift was significant. In conclusion, one can say that the developed model is robust and can achieve good results on challenging real-world PHM tasks.