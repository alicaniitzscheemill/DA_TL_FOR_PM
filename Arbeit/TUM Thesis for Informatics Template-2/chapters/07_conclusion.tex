\chapter{Conclusion}\label{chapter:conclusion}

In this thesis, a deep learning-based PHM system for monitoring BSDs was successfully extended with a domain adaptation module. A novel way of applying the MMD-loss in the layers of the CNN was proposed, which reduced the domain discrepancy efficiently. The proposed MMD-loss outperformed those restricted to the task-specific layers, which were mainly used in the literature. This was reflected primarily in the increased prediction accuracy and training stability but was also seen in the reduced sensibility to different signal and GAMMA choices. The development of a novel labeled MMD-loss revealed the main deficit of the regular unlabeled MMD-loss, which equally reduces the domain discrepancy between samples of the same and different classes. If the unlabeled MMD-loss became too dominant, the class separability in both domains was reduced, which made the classification task more challenging. Only when balancing the source CE- and MMD-loss properly, the domain discrepancy was reduced while maintaining or increasing the compactness and separability of the classes. For this reason, the GAMMA choice was identified as highly relevant for the PHM performance. It is suggested to pick the GAMMA precisely and individually for each signal.

The developed models were evaluated on a dataset recorded on an industrial machine. Therefore, the mutual influence of different components of the machine was considered in the model evaluation. Additionally, the degradation of the BSDs and LGSs was caused by field use. In summary, the evaluation of the proposed PHM system was more realistic and further elaborated than in the works presented in chapter \ref{chapter:related_works}. Furthermore, the developed model was able to predict the health condition of the BSDs independent of the LGS degradation. Compared to Pandhare et al. \cite{Pandhare2021} and Azamfar et al. \cite{AZAMFAR2020103932}, different sets of BSDs were used in the training and testing data. Since the preload class definitions differed for both domains, the domain shift was significant. Moreover, Pandhare et al. \cite{Pandhare2021} and Azamfar et al. \cite{AZAMFAR2020103932} restricted their system to monitor the preload level of BSDs. The proposed system in this thesis learned patterns to monitor different degradation types. Simultaneously, pitting damages and preload levels were predicted by the system. In conclusion, one can say that the developed model is robust and can achieve good results on challenging real-world PHM tasks.