% !TeX root = ../main.tex
% Add the above to each chapter to make compiling the PDF easier in some editors.

\chapter{Related Works}\label{chapter:related_works}


\section{Traditional Approaches for Prognostic and Health Management}

Traditionally, model-based and data-driven models were used as PHM systems. Model-based methods predict the health condition based on physical models, which describe the underlying mechanisms of degradation. Uncertainties in the machine processes as well as noise make the application of physical models difficult. Often, the identification of all model parameters is difficult and requires a lot of experiments. Data-driven methods learn a mapping relationship between the health condition state and monitoring data. Such methods do not use physical information about the degradation process. The performance of data-driven approaches highly depends on the quality and amount of the training data. Besides that, data-driven models might lack from generality. The models are just optimized to perform well on the working conditions which are present in the train data. Both, model-based and data-driven approaches suffer from different limitations \cite{DENG2020}. In the following, data-driven and model-based approaches in the context of PHM for BSDs are presented. 

\subsection{Model-Based Approach: Defect Frequency Estimation Based on a Model Simplification of Ball Screw Drives as Rolling Bearings}
Lee et al \cite{Lee2015} propose a diagnosis system, which in first place determines the characteristic frequencies for different machine defects. By filtering the machine signals for those defect frequencies, the type, severity and location of degradation signs in the machine can be predicted. Lee et al developed a test bed containing one BSD and linear motion guides. An accelerometer was mounted on the BSD nut measures vibrations. Holes with a diameter of 3 mm were punched in the BSD shaft to simulate the degradation. The ongoing process of fatigue was modeled by increasing the number of holes in the shaft. The motor was actuated with a constant velocity while the data was recorded.

Lee et al use a method, which was proposed by Harris and McCool \cite{Harris1996}, to estimate the characteristic defect frequencies. This method treats the BSD as a rolling bearing, where the screw shaft can be considered as inner and the nut as outer ring. From the construction details and the relative speed between the BSD components, the defect frequencies can be calculated. The ball pass frequencies of the shaft (BPFS) and nut (BPFN), as well as the ball spin frequency (BSF) are considered as defect frequency: 
\begin{equation}
    BPFS = \frac{1}{120}zn(1+\frac{D_{w}}{d_{m}}cos\alpha),
    \label{eq:defect_frequency}
\end{equation}
\begin{equation}
    BPFN = \frac{1}{120}zn(1-\frac{D_{w}}{d_{m}}cos\alpha),
\end{equation}
\begin{equation}
    BSF = \frac{1}{120}n\frac{d_{m}}{D_{w}} (1-\frac{D_{w}}{d_{m}}cos\alpha)(1+\frac{D_{w}}{d_{m}}cos\alpha) ,
\end{equation}
where $\alpha$ is the contact angle between the ball, nuts and screw shaft, $d_{m}$ is the pitch diameter of the balls, $D_{w}$ is the diameter of a single ball, the rotational speeds of external and internal races are defined by $n_{e}$ and $n_{i}$ and $z$ is the number of steel balls. A more detailed visualization of the bearing parameters is shown in fig. \ref{fig:defect_frequency_calc}. 

\begin{figure}[H]
  \centering
  \includegraphics[width=.6\textwidth]{models_state_of_the_art/defect_frequency_calc.pdf}
  \caption{Simplification of BSDs as bearings for frequency calculation \cite{Lee2015}}
  \label{fig:defect_frequency_calc}
\end{figure}

The above derived frequencies are valid for bearings. To apply those to BSDs one has to replace $z$ and $d_{m}$ by the effective number of steel balls $z^{'}$ and effective pitch parameter $d_{m}^{'}$, which are defined as following:

\begin{equation}
    d_{m}^{'} = (L_{p}^{2}+(\pi D_{b})^{2})^{\frac{1}{2}},
\end{equation}
\begin{equation}
    z^{'} = \frac{d_{m}^{'}}{D_{w}}.
\end{equation}
The translation between the regular and effective parameters is visualized in fig. \ref{fig:defect_frequency_transfer}. 

\begin{figure}[H]
  \centering
  \includegraphics[width=.7\textwidth]{models_state_of_the_art/defect_frequency_transfer.pdf}
  \caption{Transfering pitch parameter and number of steel balls from bearings to BSDs \cite{Lee2015}}
  \label{fig:defect_frequency_transfer}
\end{figure}

Li et al identified the BPFS frequency as the most expressive and reliable one for supervising the health condition of BSDs. To calculate the BPFS for ball screw drives the equation \ref{eq:defect_frequency} must be combined with the effective pitch parameter $d_{m}^{'}$ and effective number of steel balls $z^{'}$. The wavelet transform (Daubechies Wavelet (db14) function) is used to transform the machine signals in the two-dimensional time-frequency domain. Each time the steel balls of the BSD pass a whole in the surface of the BSD shaft, the machine signal shows a corresponding defect pattern. This time-related information are an indicator for detect location. The frequency-related information provide information about the severity and type of degradation. Fig. \ref{fig:defect_frequency_model} visualizes the proposed approach during testing.

\begin{figure}[H]
  \centering
  \includegraphics[width=.95\textwidth]{models_state_of_the_art/defect_frequency_model.pdf}
  \caption{Failure diagnosis system by calculating defect frequencies based on \cite{Lee2015}}
  \label{fig:defect_frequency_model}
\end{figure}

\subsubsection{Conclusion}
Lee et al. assume that defects and degradation are mainly subjected to rolling friction. Often times such simplifying assumptions are made when developing model-based PHM systems. In reality the balls in BSDs are rotating, revolving, and sliding, which might create a highly complex degradation process which will be hard to modeled accurately. When applying data-driven PHM systems one can define the degradation classes individually by labeling the recorded dataset accordingly. Lee et al. have to stick to the definitions set by Harris et al. \cite{Harris1996}, when they defined defect frequencies corresponding to specific physical degradation mechanisms in rolling bearings. When searching machine signals for the specific defect frequencies Lee et al. came across other distinct frequencies coming from major electrical noise. Lee et al. just tested their approach on a small testbed. When applying that approach on real-world machines, vibrations from other components might contaminate the recorded signal with unwanted noise. Due to the limited accessibility for sensor installation on the ball nut, the direct measurement of the nut vibration might be difficult. Finding all parameters to calculate the defect frequencies might require a lot of effort in measuring and testing. Often these parameters are assumed to be constant throughout the lifetime of the BSD. The rolling diameter for example will be reduced thoughout the lifetime of a rolling bearing. Anhow, this parameter is used as a constant parameter in the calculation of the defect frequencies for rolling bearings. When transfering the calculated defect frequencies from the rolling bearing to the BSD only structural differences are considered. Lee et al. completely ignore the linear movement, which differs the function of rolling bearings and BSDs fundamentally. Lee et al. simulated the degradation process by punching holes with a diameter of 3 mm on the grooves of the BSD shaft. Solely the number of holes, but not their size does change during the simulation of the degradation. It is questionable whether this method simulates the degradation appropriately.  

\subsection{Model-Based Approach: Discrete Dynamic Modeling}
Nguyen et al. \cite{NGUYEN2019} derived a simplified dynamic model, which is shown in fig. \ref{fig:Nguyen_discrete_dynamic_model}, to investigate the relations between BSD preload variations and the axial natural frequency.

\begin{figure}[H]
  \centering
  \includegraphics[width=1\textwidth]{models_state_of_the_art/Nguyen_discrete_dynamic_model.pdf}
  \caption{discrete dynamic model \cite{NGUYEN2019}}
  \label{fig:Nguyen_discrete_dynamic_model}
\end{figure}

A formula was developed which relates the preload variations with the variation in the stiffness of the screw nut: 

\begin{equation}
    k_{nut}=0.8K(\frac{P}{0.1C_{a}})^{\frac{1}{3}},
\end{equation}

where $P$ is the BSD preload, $C_{a}$ is the BSD screw dynamic load, $K$ is the BSD nut stiffness according to the manufacturer and $k_{nut}$ is the actual changing BSD nut stiffness. The forula does just valid if the BSD preload is less than 10\% of the BSD dynamic load. The axial and rotational stiffness of the BSD shaft is defined as following:
\begin{equation}
    k_{shaft}^{ax}=\frac{EA}{x_{t}}=\frac{\pi}{4x_{t}}d_{minor}^{2}E,
\end{equation}
\begin{equation}
    k_{shaft}^{rot}=\frac{\pi}{32L}d_{minor}^{4}G,
\end{equation}
 where $A$ is the cross sectional area of the BSD shaft, $E$ the Young’s modulus, $d_{minor}$ the screw diameter, $G$ the shear modulus, $L$ the screw length and $x_{t}$ the working table displacement. The total BSD axial stiffness of the ball screw is composed of the stiffness of the screw shaft, supporting bearing, screw nut, and bracket:
 \begin{equation}
    \frac{1}{k_{ax}}=\frac{1}{k_{shaft}^{ax}}+\frac{1}{k_{bearing}^{ax}}+\frac{1}{k_{nut}^{ax}}+\frac{1}{k_{bracket}^{ax}}+\frac{1}{\frac{k_{shaft}^{rot}}{\beta^{2}}}.
\end{equation}
The total rotational rigidity is estimated from the rotational BSD stiffness and the stiffness of the flexible coupling between the motor and BSD shaft:
 \begin{equation}
    \frac{1}{k_{rot}}=\frac{1}{k_{shaft}^{rot}}+\frac{1}{k_{coupling}}.
\end{equation}
From there the axial natural frequency is calculated in relation to the axial stiffness and mass of the ball screw system:
\begin{equation}
    f\approx\frac{1}{2\pi}\sqrt{\frac{k_{ax}}{m_{table}+m_{screw}+m_{nut}+m_{bracket}}}=\frac{1}{2\pi}\sqrt{\frac{k_{ax}}{\sum M}}.
\end{equation}

The axial natural frequency increases with the preload and decreased with the mass of the working table. The BSD preload can be modeled based on the axial natural frequency, mass and and displacement of the ball screw system and some earlier specified fixed system parameters:
\begin{equation}
    P=\frac{0.1C_{a}}{\{0.8K[ -\frac{4x_{t}}{\pi d_{minor}^{2}E} -\frac{32\pi^{2}L}{\pi d_{minor}^{4}G}-\frac{1}{k_{bearing}}-\frac{1}{k_{bracket}}+\frac{1}{(2\pi f)^{2}\sum M} ]\}^{3}}
\end{equation} \label{eq:preload_estimation_based_natural_frequency}
A monitoring program was developed, which receives and processes the signals, composes the frequency response function (FRF), estimates the axial natural frequency and predicts the BSD preload based on equation \ref{eq:preload_estimation_based_natural_frequency}. When the estimated preload is below a given threshold, an alarm is set. During the inspection of the BSD the table mass must be kept constant. An encoder included in the motor controller is used to estimate displacement of the working table. The FRF is composed from the vibration signal measured with an uni-axial acceleration sensor mounted ad the screw nut and the motor current signal estimated by three Hall-effect based current sensors. Especially when the motor speed changes rapidly, the modes of the BSD system are activated strongly. In these phases the deformation of the BSD system is strong enough to show critical information about the BSD modes and is distinguishable from the vibration of other components. A trigger is installed to detect the phases of rapid motor speed change. During those phases, the vibration and motor current signal are recorded, widowed and FFT transformed. The resulting auto-spectrum and cross-spectrum are used to compute the FRF. By computing and averaging the FRF computed at several occurences at the same positions along the BSD shaft more stable results were achieved. The frequency in the FRF with maximum magnitude is the axial natural frequency, which is found by a peak detection algorithm \cite{NGUYEN2019}. The calculation of the averaged FRF is again visualized in fig. \ref{fig:Nguyen_frf}

\begin{figure}[H]
  \centering
  \includegraphics[width=.8\textwidth]{models_state_of_the_art/Nguyen_FRF.pdf}
  \caption{Average FRF calculation \cite{NGUYEN2019}}
  \label{fig:Nguyen_frf}
\end{figure}

\subsubsection{Conclusion}
Nguyen et al. \cite{NGUYEN2019} developed a complex inspection procedure to predict the preload of BSDs. The discrete dynamic model presented in fig. \ref{fig:Nguyen_discrete_dynamic_model} is the base of the preload prediction. The BSD system is simplified by springs and dampers. The preload prediction highly depends on the accuracy of the model and its closeness to the real-world BSD. The modes of the BSD system are activated just in extreme cases, where the motor changes its velocity rapidly. Just in these cases the BSD modes are dominant enough to become visible and not being disturbed by the vibrations of other components. The proposed inspection of the BSDs takes place in a very supervised and isolated test bed. When analyzing BSDs, which are installed in big industrial machines, finding the axial natural frequency might become more challenging. The BSD modes might be less distinguishable from the vibration of other components. The goal of such monitoring systems is the continuous estimation of the BSD preload for long time horizons. Operational conditions of the industrial machines and the weight of the working table might not stay constant throughout those time horizons. The relationship between the axial natural frequency, the displacement of the working table and the BSD preload might then not be reliable anymore. This could lead to unsatisfactory prediction performance of the monitoring system. Even in the proposed isolated BSD test bed, the trigger sometimes failed to detect expressive operational phases for estimating the axial natural frequency. Implementing such a trigger for the BSDs in big industrial machines might be even more difficult. Nguyen et al. emphasize that at least 10 trigger events from each position during the operation is required to average the FRF in practical applications. Collecting this big amount of trigger events involves a lot of effort. 

\subsection{Data-Driven Approach: Principal Component Analysis based Sensor Fusion of Multiple Statistical Features}

Denkena et al. \cite{Denkena2021} present a method to monitor the degradation of BSDs. Firstly, several hand-crafted statistical features are extracted from the machine signals. Afterwards, a sensor fusion approach based on principal component analysis is used to combine those features. In the end, decision trees solve the classification task. A test bed consisting of a single BSD was used to evaluate the proposed approach. Internal control signals were provided by the internal controller and three uniaxial acceleration sensors measure vibrations in the system. The machine data was recorded during constant feed rate over the whole length of the test bench. The proposed PHM method can be separated in four steps:

\begin{itemize}
    \item [\textbf{Data acquisition:}] The method processes vibration signals from three uniaxial acceleration sensors as well as internal control signals simultaneously. The signals are concatenated, synchronized and separated in phases of zero acceleration (constant movement of nut on the shaft) and non-zero acceleration (direction change movement at each end of the BSD shaft). The ball screw was moved over the length of the test bench with various feed rates (6000 mm/min, 11000 mm/min, 17000 mm/min,20000 mm/min) \cite{Denkena2021}.
    \item [\textbf{Feature extraction:}] Information about the preload classes are extracted through statistical features (e.g. kurtosis, median, impulse factor, ...). The features are extracted for each signal and segment. Each feature is evaluated by its robustness and statistical significance. The robustness is measured by the feature's dispersion around the median. After normalizing the feature with the z-score, the significance is investigated by the f-statistics.
    
    \begin{equation}
        \textbf{Z-score:}\qquad \tilde{x}_{i,j} = \frac{x_{i,j} - \bar x_{i}}{\sigma_{i}},
    \end{equation}
    
    \begin{equation}
        \textbf{F-statistic:}\qquad f = \frac{\sum_{j=1}^{J} i \cdot (\bar x_{j} -\bar x)^{2}/(J-1)}{\sum_{j=1}^{J} \sum_{i=1}^{I} i \cdot (\bar x_{j,i} -\bar x_{j})^{2}/(J \cdot (I-1))},
    \end{equation}
    where ${x}_{i,j}$ denotes the feature value j of a sample of class i, $\bar{x}_{i}$ is the feature mean value of class i, ${s}_{i}$ is the feature standard deviation of class i and $\bar{x}$ is the overall feature and class mean value, I and J is the number of all classes and features. Denkena et al. see features as eligible for the diagnosing system if the dispersion around the median is smaller than $\pm$ 10 and the f-statistics is higher than a critical value of 10. The selected features are merged with the goal of maintaining the robustness and increasing the f-statistics  \cite{Denkena2021}. 
    
    \item [\textbf{Principal Component Ananylsis:}] 
    Principal component analysis (PCA) is a method to reduce the dimensionality of a data while retaining most of the information. Principle components are the directions in the feature space, along which the variation of the data is maximal. Using just a few principal components, each sample can be represented by a few number of variables \cite{Ringner2008}.
    
    \item [\textbf{Classification:}] Denkena et al. decided to use a decision trees to predict a preload class based on the extracted features. Due to its low classification effort and good traceability, decision trees seem suitable for the classification task. For each signal and segment, a separate decision tree is used  \cite{Denkena2021}. 
\end{itemize}

\subsubsection{Conclusion}
In order to find features which work well for predicting the degradation state of BSDs, Denkena et al. extracted 1500 features from the signal segments. For feed rates higher than 11000 mm/min, the acceleration signals are not able to classify the preload classes accurately. The univariate features MAX, RMS and CRE can only separate some of the defined preload class. Depending on the signals, the extracted statistical features showed sufficient robustness and statistical significance. This shows, that the suitability of the features depend strongly on the working conditions, classes and signals. This rises question about the applicability of the proposed approach for the real-world use on industrial machines. The features do not show a proportional behaviour to the physical degradation. When applying the features MAX and RMS on the acceleration signals, they increase from C3 through C2 to C1, but decrease towards C0. Often deep learning is criticized for its black box principle and the lack of understanding for the extracted features. Even though one generally understands the hand-crafted features, the meaningfulness of the extracted information depends a lot on the task and signal. The correlation between those features and the physical degradation is often highly unknown, which makes such features an equal black box as features extracted by deep learning models. Dekena et al. mention, that slower feed rates improve the classification performance of some features. Anyhow, one has to remember that those feed rates do also increases the inspection time . Generally, one can say that finding a set of features which reliably predicts all BSD preload classes demands quite some effort. These features just work reliably for specific signals, working conditions and defined preload classes. If any of those parameters change, the whole feature analysis has to be repeated. Denkena et al. simulate the preload loss throughout the lifetime of BSDs, ball sets with different oversize were installed in the BSD. The physical degradation of the shaft the balls is neglected in the experiments.

\subsection{Data-Driven Approach: Multi-Level Feature Selection Module for Health Diagnosis, Assessment and RUL Prediction}
Li et al. \cite{LiPin2018} developed a prognosis system for BSDs, which does fault diagnosis, early diagnosis, health assessment and remaining useful life (RUL) prediction simultaneously. Since this thesis focuses on the fault diagnosis of BSDs, this chapter is restricted  to those processing units that are related to that. Li et al. build a testbed containing a BSD and horizontal guideways fixed to a concrete base. Instead of manually inserting faults to simulate degradation, it was caused by field use. Fig.  \ref{fig:level_feature_selection_model} shows the different processing steps and the parallel prediction units for the fault diagnosis, health assessment and remaining useful life (RUL) prediction. Vibration data is measured by three accelerometers and speed and torque signals are retrieved from the controller \cite{LiPin2018}. 

\begin{figure}[H]
  \centering
  \includegraphics[width=1\textwidth]{models_state_of_the_art/model_multi-level_feature_selection.pdf}
  \caption{System architecture for multi-level feature selection module for health diagnosis, assessment and RUL prediction proposed by Li et al \cite{LiPin2018}}
  \label{fig:level_feature_selection_model}
\end{figure}

\begin{itemize}
    \item [\textbf{Feature Extraction}]: Each signal is transformed in the time domain, frequency domain and the time frequency domain. The features are applied on all three domains. Similary to Denkena et al. \cite{Denkena2021}, summary statistics, like RMS, mean, variance, Kurtosis and Skewness, are exctracted from the signals. Wavelet decomposition (‘db4’ wavelet) is performed on each signals to extract the energy level from different frequency bands. Like in the work of Lee et al \cite{Lee2015} the amplitudes corresponding to the calculated ball passing frequencies, as well as ball screw rotation frequency and its harmonics are measured \cite{LiPin2018}.
    \item [\textbf{Feature Selection}]: In a multi-level feature selection procedure the extracted are rated based on their suitability for the prognosis task. This process is separated in three stages: primary feature ranking, pre-selection to find useful features and fine-selection to extract optimal features. In order tp select expressive features, Li et al. \cite{LiPin2018} propose a hybrid strategy of filter-based and the wrapper based approaches. A selection criterion first ranks and filters all extracted features which achieve higher scores than a specified threshold. Those useful features are then processed wrapper-based pre-selection approach. The searching space are the previously selected useful features and the search sequence corresponding ranking score defined by the selection criterion. In the proposed approach the fisher score is chosen for the pre-selection:
    \begin{equation}
        S_{c} = \frac{\sum_{k=1}^{C} n_{k}(\mu_{k}^{j}-\mu^{j})^{2}}{\sum_{k=1}^{C}n_{k}(\sigma_{k}^{j})^{2}}
    \end{equation}
    For the multi-class classification task the fished score is biased. Therefore a final feature-selection mechanism is added subsequently. An SVM is applied and based on the prediction accuracy the final optimal features are selected \cite{LiPin2018}.
\end{itemize}

\subsubsection{Conclusion}
From the raw data Li et al. extracts 440 features. In addition three levels of feature selection and corresponding empirical thresholds are defined. First of all, finding suitable features and selection mechanism already expects a huge effort. In addition, all these design choices are very specific and depend highly on the working conditions, defined health classes and signals. When for example looking at the vibration signals, one can observe that the expressiveness of the features differ hugely. Also, when applying the same features to the vibration and torque signals, a performance of the features partially diverged strongly. Besides that, Li et al. struggled to find a good empirical threshold for the fisher criterion for the 9 class classification task. Just for the binary classification the threshold was known to be 1. The just mentioned three problems show the high dependency of such approaches on the predefined conditions of the PHM task. Minor variations will most certainly reduce the system performance or make it even fail completely. The SVM-based fine-selection strongly changed the proposed feature ranking from the fisher score. Especially the detected top features were rated lower during the dine-selection. This adds an additional design choice, where one has to rate the significance of the individual feature selection mechanisms. Using SVMs in the feature selection adds additional training effort in the  system. This increases time and data required for the optimization of the proposed system. Similarly to the work of Denkena et al. \cite{Denkena2021}, some features did not correlate well with the degradation process. 

\section{Domain Adaptation Approaches for Prognostic and Health Management of Ball Screw Drives}
In recent years, more and more intelligent and adaptive data-driven approaches were proposed for monitoring the health status of industrial machines. Especially in the computer vision community, domain adaption and transfer learning became a hot topic. Slowly models, developed in the computer vision context, also make its way in the PHM field. In the following, deep learning-based domain adaption approaches are presented for predicting the degradation status of BSDs and rolling bearings. Similarly to the method proposed in this thesis, the presented models are optimized to reduce the domain discrepancy measured by the MMD metric.


\subsection{Deep Domain Adaption based on MMD-Loss}
Azamfar et al \cite{AZAMFAR2020103932} proposed a deep learning-based domain adaption approach for PHM of BSDs, using a MMD-loss. An experimental test rig was build, containing a single horizontal guideway and a BSD, which were both fixed on a concrete base. Three accelerometers were installed to measure vibrations in X and Y directions. These sensors were mounted on the BSD nut and the bottom and top attachments of the BSD shaft. A sound pressure sensor captured the acoustic level when running experiments on the test rig. The torque and speed signals were acquired from the controller. In the experiments the guideways and BSDs were available in three different degradation classes. In the "normal" class the concerned component was operating normally, in the "faulty level 1" class it was deviating from the healthy condition and in the "faulty level 2" class it had to be replaced or repaired. In total nine different combinations of guideways and BSDs with different levels of degradation were combined. Azamfar et al acquired data by performing a full cycle of BSD operation, which contains two full forward and backward movements along the guideways. The data from each recording was split in phases with constant (forward, backward movement) and changing BSD nut velocity (turning point at the end of BSD shaft). The training was restricted to the signals recorded during the phases with constant speed, which were fed to the model as a single input without any windowing. The data dimension was reduced by a simple down-sampling method. The full cycle of BSD operation was recorded with different BSD velocities (200, 400 and 600 mm/s). The data distribution discrepancy in the recordings with different BSD velocities created a domain shift. The proposed method was evaluated on a 9 class classification task, including all combinations of BSD and guideway degradation classes. The proposed neural network architecture is presented in fig. \ref{fig:Azamfar_model}. It contains a feature extractor of four alternating one-dimensional convolutional and max-pooling layers and a subsequent classifier. To prevent overfitting, the dropout layers with the rate of 0.3 are included. The ReLU activation function is used throughout the network. The proposed model optimization includes a source CE-loss to improve the classification accuracy on the source domain data. Besides that the domain discrepancy is reduced by a MMD loss, which is applied in the penultimate fully connected layer. 

\begin{figure}[H]
  \centering
  \includegraphics[width=1\textwidth]{models_state_of_the_art/Azamfar_model.pdf}
  \caption{Architecture proposed by Azamfar et al \cite{AZAMFAR2020103932}}
  \label{fig:Azamfar_model}
\end{figure}


\subsection{Deep Domain Adaption based on MMD-Loss and PD Alignment}
Pandhare et al. \cite{Pandhare2021} proposed a deep learning-based domain adaption approach for PHM of BSDs, using a MMD-loss and PD alignment. A similar test rig, as the one presented by Azamfar et al \cite{AZAMFAR2020103932}, was used by Pandhare et al to evaluate the proposed models. In total, five accelerometers were mounted in the test bed. Two triaxial ones were placed close to the BSD nut, which seems a promising position to represent the signature of a ball screw preload level. Three single-axial ones were mounted at the bottom and top attachments of the BSD shaft and on top of the load carried by the BSD nut. The last three sensor positions are more suitable and practical installation. Identical to Azamfar et al \cite{AZAMFAR2020103932} nine combinations of degradation classes of BSD and guideways were defined. The domain discrepancy between datasets generated with differently mounted sensors should be reduced by the domain adaption approach of the proposed model. The work tries to find an indirect sensing method to make PHM on BSDs independent of impractical sensor locations. The proposed model architecture is presented in fig. \ref{fig:Pandhare_model} and contains a feature extractor of two convolutional and max pooling layers and a consecutive classifier. The proposed model training includes three losses. Again, a source CE-loss is used to improve the classification performance on the source domain data. The MMD loss reduces the marginal distribution mismatch between the domains. The PD alignment reduces the conditional distribution discrepancy of the domains. This is done by matching source and target samples with equal class label and reducing their L2-distance: 

\begin{equation}
    L_{p} = \frac{1}{n_{p}}\sum_{k=1}^{n_{p}}|h_{k}^{p,s}-h_{k}^{p,t}|_{2}, 
\end{equation}
where $h_{k}^{p,s}$ and $h_{k}^{p,t}$ are the k-th source and target domain samples and $n_{p}$ is subspace of the labeled samples from source and target. Pandhare et al. did restrict the PD-alignment just on some of the 9 classes and used a ratio of 20\% between training samples used in the CE and MMD loss and PD samples. 

\begin{figure}[H]
  \centering
  \includegraphics[width=.6\textwidth]{models_state_of_the_art/Pandhare_model.pdf}
  \caption{Architecture proposed by Pandhare et al \cite{Pandhare2021}}
  \label{fig:Pandhare_model}
\end{figure}

\subsection{Conclusion}
Since the presented approaches by Azamfar et al. \cite{AZAMFAR2020103932} and Pandhare et al. \cite{Pandhare2021} have similar problems and disadvantages, they will both be presented in this section. First of all, both approaches apply a regime separation to separate the signals in phases of constant and changing BSD velocity. This extra step adds additional complexity to the data pre-processing. Both approaches avoid the windowing of the signals. The segments with constant BSD velocity generated by the regime separation are fed to the models as single samples. Both works assume, that the singal of the full BSD stroke is more suitable for the monitoring task since it captures the frequenciy and amplitude variations during the steady-state phase of the BSD movements. This might be correct and simplify the task but also demands a lot of experiments and recordings in order to generate a dataset big enough to properly train neural networks. Azamfar et al. and  Pandhare et al. evaluate their monitoring approaches solely on a simplified test bed, where the BSD module runs along two horizontal guideways and the whole mechanism is fixed to the concrete. The evaluation of the approaches does not consider the mutual influence of all the components in industrial machines. Besides that, both approaches are restricted to predict the current preload in the BSDs. Other damages like pitting were ignored by the monitoring systems. In both cases the MMD loss was just applied in the last fully connected layer without evaluating other latent feature spaces for their suitability to reduce the domain discrepancy. Azamfar et al. dealt with a domain shift created by recording datasets with different BSD velocities. Pandhare et al. used different sensors to record data. A domain shift was created by recording data with accelerometers mounted at different positions in the BSD testbed. In both cases the domain shift does not include any differences on the physical component level. The components of interest, whose degradation level should be predicted by the monitoring system were the same for the source and target domain. The dmain adption modules do just have to deal with domain shifts coming from variations in the operational conditions and recording of the data. It was not evaluated how the presented approaches react to physical variations in the systems. The PD-alignement approach presented by Pandhare et al. expects the target class labels from some of the considered classes, which simplifies the health monitoring task


\section{Domain Adaptation Approaches for Prognostic and Health Management of Rolling Bearings}



\subsection{Deep Domain Adaption based on MMD-Loss Deep Distance Metric Learning}
A PHM algorithm for rolling bearings, which optimizes the inter- and intra-class distance in the latent feature space and reduces the domain discrepancy with a MMD-loss, is presented by Li et al \cite{Li2018}. As visualized in fig. \ref{fig:Deep_distance_metric_learning_model}, the model proposed by Li et al contains a CNN and a consecutive classifier. In a preprocessing step, a FFT transform is applied to represent the raw vibration signals in the time-frequency domain. Max-pooling layers are included to reduce the model size. Batch-normalization layers reduce the internal covariate shift by normalizing the input distributions of the hidden layers. To prevent overfitting, dropout layers with a rate of 0.5 are included. The proposed method was evaluated on a rolling bearing dataset provided by the Bearing Data Center of Case Western Reserve University. The domain shift was generated by using testing data, which was exposed to additional environmental noise, and collected under a different working conditions than the source data. 10 health conditions with different fault location and fault size were defined. 

\begin{figure}[H]
  \centering
  \includegraphics[width=.75\textwidth]{models_state_of_the_art/Deep_distance_metric_learning_model.pdf}
  \caption{Deep distance metric learning architecture proposed by Li et al \cite{Li2018}}
  \label{fig:Deep_distance_metric_learning_model}
\end{figure}

Li et al suggest to optimize the model such that the distance between source samples is minimized if they belong to the same class and maximized otherwise. This increases the separability and compactness of the classes, which makes the algorithm more robust against environmental noise. In order to calculate the intra- and inter-class distances the expectation and variance of source domain samples belonging to the same class is measured in the feature maps:

\begin{equation}
    \begin{aligned}
       &D_{inter} = |E[f^{(m)}x^{(i)}]-E[f^{(m)}x^{(j)}]|_{2}-\sqrt{Var[f^{(m)}x^{(i)}]}-\sqrt{Var[f^{(m)}x^{(j)}]}\\
       &D_{intra} = 
        \sum_{i=1}^{N_{class}} \sqrt{Var[x^{(i)}]},
    \end{aligned}
\end{equation}

where $N_{class}$ is the number of classes, $x^{(k)}$ denotes the raw input sample of class k, $f^{(m)}x^{(k)}$ denotes the feature representation of this sample in the m-th layer and $E[f^{(m)}x^{(i)}]$ and $Var[f^{(m)}x^{(i)}]$ are the corresponding expectation and variance. The inter- and intra-class distance are optimized with the following loss: $J_{Cluster} = - D_{inter} + \eta D_{intra}$. Since $J_{Cluster}$ requires the sample's labels, the optimization is restricted to the source domain. In addition to the distance metric learning the domain discrepancy is reduced by an MMD-loss in several FC layers. Lastly, a CE-loss in the final layer optimizes the network to classify the source samples correctly. In total, the network is optimized with the following weighted average of losses: 

\begin{equation}
    \begin{aligned}
    J_{total} = \alpha J_{Cluster} + \beta J_{MMD} + \gamma J_{CE}, 
    \end{aligned}
\end{equation}
where $J_{Cluster}$ is the loss optimizing the distances between the source domain samples, $J_{MMD}$ the MMD-loss,  $J_{CE}$ the CE-loss and $\alpha$, $\beta$ and $\gamma$ are the weights for calculating the weighted average \cite{Li2018}.

\subsection{Conclusion}
There are several domain adaption approaches for PHM of rolling bearing based on MMD-losses \cite{AN201942} \cite{Li2018} \cite{Guo2019} \cite{Singh2019} \cite{Kang2020}. Generally, BSDs and rolling bearing are related components. BSD shafts can be seen as the inner and BSD nuts as the outer ring of rolling bearings. In both cases balls between those two components allow a rotatory motion around a fix axis. Other than rolling bearings, BSDs also translate this rotatory motion in a linear motion. The degradation of rolling bearings and BSDs is related in some sense. Nevertheless, bearing PHM applications cannot be relied to work as well for BSDs. Still the research in this domain offers interesting applications and details for the PHM of BSDs. 




\section{Computer Vision Applications which Reduce the Domain Discrepancy in the Feature Extractor}
Most domain adaption approaches reduce the domain discrepancy in task-specific layers but use a shared feature extractor backbone across all domains. In the following two computer vision models are presented, which reduce the domain shift in early layers of the feature extractor. Often the computer vision community offers advanced solution for complex research questions, which were not intensively evaluated in real-world scenarios. For the PHM of BSDs such solutions can be taken as inspiration.

\subsection{Domain Conditioned Adaption Network}
Li et al \cite{li2020} propose a domain conditioned adaption network (DCAN), which contains two separate modules to reduce the domain discrepancy. After each task-specific layer a domain conditioned feature correction block estimates and reduces the domain discrepancy based on the MMD metric. In the CNN backbone an attention module regulates the extraction of domain-specific and -independent features. The proportions of domain-specific and -independent features can be learned to decrease the domain discrepancy. Fig. \ref{fig:DCAN_model} visualizes the domain adaption modules in the DCAN model.

\subsubsection{Domain Conditioned Channel Attention Mechanism}
ResNet is used as backbone network, which allows an easy implementation of the domain conditioned channel attention module in its residual block. In the latent feature maps the processed images are represented as $\pmb{X}_{t} = [X^{1}_{t},...,X^{C}_{t}] \in \mathbb{R}^{HxWxC}$, where H and W are the spatial dimension and C the number of image channels. First, a channel-wise global average pooling layer is applied, which reduces the images to  $\pmb{g}_{t} = [g^{1}_{t},...,g^{C}_{t}] \in \mathbb{R}^{1x1xC}$. Afterwards, the data is split depending on its domain and passed through different fully connected layers. The upper flow is used for target and the lower flow for source domain samples. The two different source and target domain routes share parameters. For both domains, an attention mechanism is trained jointly to learn to activate different channels in the domain samples. This allows extracting more enriched domain-specific features. In the fully connected layers the dimension is first reduced with a ratio ${1x1x\frac{C}{r}}$ and later reconstructed to its original size ${1x1xC}$. ReLU and Sigmoid functions are applied. The domain-wise feature selection is achieved by weighting the channels of the feature representations $\pmb{X}_{s}$ and $\pmb{X}_{t}$ with the channel attention vectors $\pmb{v}_{s}$ and $\pmb{v}_{t}$ calculated by the domain conditioned channel attention module:

\begin{equation}
    \begin{aligned}
        &\pmb{\tilde{X}}_{s} = \pmb{v}_{s} \odot \pmb{X}_{s} = [v_{s}^{1} \cdot X_{s}^{1}, ..., v_{s}^{C} \cdot X_{s}^{C}]\\
        &\pmb{\tilde{X}}_{t} = \pmb{v}_{t} \odot \pmb{X}_{t} = [v_{t}^{1} \cdot X_{t}^{1}, ..., v_{t}^{C} \cdot X_{t}^{C}].
    \end{aligned}
\end{equation}

The domain conditioned channel attention module allows the model to independently learn the importance of each channel for the classification of source and target domain samples \cite{li2020}.


\subsubsection{Domain Conditioned Feature Correction}
The data simultaneously passes through the regular network and the feature correction block, which consist of FC and ReLU blocks. The feature correction block estimates the domain discrepancy in the feature representation of the task-specific layer:
\begin{equation}
    \Delta H_{l}(x_{t}) = H_{l}(x_{s}) - H_{l}(x_{t}),
\end{equation}
where $H_{l}(x_{s})$ and $H_{l}(x_{t})$ are the feature representations of the source and target domain samples in the task-specific layer l. $\pmb{x}_{s}$ and $\pmb{x}_{t}$ are the source and target domain samples. The feature representation of the target domain samples is corrected as following:

\begin{equation}
    \hat{H}_{l}(x_{t}) = H_{l}(x_{t}) + \Delta H_{l}(x_{t}).
\end{equation}

The discrepancy between the regular feature representation of source domain samples $H_{l}(x_{s})$ and the corrected feature representation of the target domain samples $\hat{H}_{l}(x_{t})$ is measured by the MMD-loss in several layers:

\begin{equation}
    L_{M}^{l} = |\frac{1}{n_s} \sum_{i=1}^{n_{s}} \phi(H_{l}(x_{si}) - \frac{1}{n_t} \sum_{i=1}^{n_{t}} \phi(\hat{H}_{l}(x_{ti}))|_{H_{\kappa}}^{2}, 
\end{equation}
where $H_{\kappa}$ is the reproducing kernel Hilbert space (RKHS), $\kappa$ the characteristic kernel and $\phi$ the corresponding feature map. The number of source and target samples is defined by $n_{s}$ and $n_{t}$. To avoid the over-transfer of noise and irrelevant information between source and target, the model is enforced to keep the source data constant when passing through the feature correction blocks \cite{li2020}.

\begin{figure}[H]
  \centering
  \includegraphics[width=1\textwidth]{models_state_of_the_art/DCAN_model.pdf}
  \caption{DCAN architecture proposed by Li et al \cite{li2020}}
  \label{fig:DCAN_model}
\end{figure}

\subsection{Feature Reconstruction of Domain Shift Affected Layers}
Aljundi et al \cite{Aljundi2016} present a method, which analyzes the output of each convolutional layer for domain shift effects. The goal is to find the layers which suffer most from those effects. By reconstructing them, the new target sample outputs should become more similar to the response given for the source samples. In a first step, the domain shift effects with respect to each layer is measured:
\begin{equation}
    B^{*} = argmin_{B} \{ \sum_{i=1}^{n}( y_{i}-\beta_{0}-\sum_{j=1}^{p}x_{i,j}\beta_{j})^{2} + \lambda \sum_{j=1}^{p}|\Delta_{j}^{KL}\cdot \beta_{j}| \}
\end{equation}
where y is the response, x the output of the various layers, $\beta_{0}$ the residual, $B = \{\beta_{j}\}$ the estimated coefficients for each layer, n the number of
source samples, p the number of layers, $\lambda$ a tuning parameter to punish the value of the coefficients and $\Delta_{j}^{KL}$ the KL divergence measured between the source and target data representation in each layer. When a layer's coefficient is big, it is considered as good (small domain shift effects) and otherwise as bad (big domain shift effects). The optimization is solved by using the coordinate descent method. Afterwards, linear regression is applied to predict new coefficients for the good layers to replace the output of the bad ones. The reconstructed layer outputs are then passed to the subsequent layers. Aljundi et al identified that especially the early layers of the feature extractor are relevant to counteract the domain shift \cite{Aljundi2016}.